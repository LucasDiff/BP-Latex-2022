% !TEX root = ../thesis.tex

\chapter{Systémová príručka}
Táto časť slúži na opis priečinkov a rozdelenia kódu.
\section*{Rozdelenie kódu}
Funkčný kód je rozdelený na dve časti. Prvá časť sa venuje základnej úprave dát do čitateľnej formy, konkrétne vo funkcii main napísanej v kóde java. Druhá časť, ktorá klasifikuje všetky dáta, spracováva výsledky a vypočítava skóre, je napísaná v Pythone, konkrétne v main funkcii predict.py. 
\section*{Opis kódu}
Prvá časť v kóde main.java má dve funkcie. Je to funkcia GetCharFromString, ktorá vracia aktuálny charakter v stringu a funkcia main, v ktorej otvoríme ako BufferedReader textový súbor 'menochampiona'+1.txt a zároveň vytvoríme nový textový súbor 'menochampiona'+.txt. Následne prechádzame celý textový súbor a upravujeme ho do čitateľnej podoby.
\\ Druhá čast v kóde predict.py má dve hlavné priority. Prvá je zápis všetkých získaných textových súborov z programu javy do jedného dataframu a zároveň ich klasifikácia. A druhá pracuje s daným dataframom a počíta skóre.
\section*{Opis priečinkov}
V priečinku transformdata sa nachádza 332 súborov, z čoho je viac než 320 textových súborov s dátami o každom jednom šampiónovi. Zároveň je tam aj hlavná časť kódu a to súbor predict.py, ktorý má v sebe spracovanie a úpravu všetkých dát. Hlavný súbor s dátami sa volá winratedata.txt ktorý v sebe drží hodnoty všetkých klasifikovaných dát pre každého šampióna. Takisto v hlavnom priečinku môžete nájsť vysledky.xlsx, čo je excelovská tabuľka všetkých otestovaných zápasov. V priečinku src nájdeme kód main.java, pomocou ktorého sme dáta dostali do čitateľnej podoby pre náš python kód. 

\chapter{Používateľská príručka}
V tejto časti si ukážeme ako spustiť a obsluhovať aktuálny kód.

\section*{Potrebné súbory}
Na úspešné odskúšanie kódu na predikciu potrebujete mať textové súbory winratedata.txt, winlose.txt, teams.txt a zároveň samotný súbor s kódom predict.py, ktorý sa nachádza v hlavnom priečinku spolu s textovými súbormi.

\section*{Príprava prostredia a knižníc} 
Odporúčané prostredie je Spyder, ideálne verzia 5.0.5 a novšie. Prostredie Spyder je možné stiahnuť na ich domácej stránke zadarmo. Potrebná knižnica na stiahnutie je pandas, čo môžete vykonať príkazom - pip install pandas, napísaným v konzole.

\section*{Popis používania}
Kód predict.py je potrebné otvoriť v predpripravenom prostredí. Následne na vrchu kódu je 5 premenných top, jg, mid, bot a sup. Pri predikcii kompozície, ktorú chcete odskúšať do týchto premenných, vložte mená šampiónov bez medzier v ich mene s malými písmenami. Ak ste tak vykonali, spustite kód zelenou šípkou a v konzole by mali byť vypísané premenné totalgames, totalscore a známka, čo označuje aktuálnu kompozíciu, ktorú ste zapísali. 
