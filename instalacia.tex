\chapter{Inštalácia typografického systému \LaTeX}
\label{ch:instalacia}

\section{Typografický systém \LaTeX}

Pre zvládnutie tohto jazyka neexistuje lepší spôsob, ako v ňom proste začať dokumenty rovno písať. Pre zvládnutie základov odporúčame použiť voľne dostupnú publikáciu \emph{The Not So Short Introduction to \LaTeX} \cite{lshort}, ktorú do slovenčiny preložili \emph{Ján Buša st.} a \emph{Ján Buša ml.} \cite{lshortsk}.

\section{Odporúčané balíčky}
\begin{itemize}
    \item \emph{minted}\footnote{\url{https://www.ctan.org/pkg/minted}} - zvýrazňovač zdrojového kódu (syntax highlighter) pre \LaTeX 
    \item \emph{europecv}\footnote{\url{https://www.ctan.org/pkg/europecv}} - šablóna pre písanie životopisov vo formáte EuroPass 
    \item \emph{PGF/TikZ}\footnote{\url{http://www.ctan.org/pkg/pgf}} - dvojica jazykov, pomocou ktorých je možné vytvárať vektorovú grafiku 
\end{itemize}

\section{Inštalácia v OS Windows}

Ak pracujete v \emph{OS Windows}, stiahnite si distribúciu \LaTeX-u s názvom \emph{TeX Live} zo stránky \url{https://www.tug.org/texlive/}.

\section{Inštalácia v OS Linux}

Ak používate distribúciu \emph{Fedora 23}, pre používanie šablóny budete potrebovať nainštalovať nasledujúce balíčky:

\begin{minted}{bash}
$ sudo dnf install texlive-bibtopic texlive-cslatex \
                 texlive-collection-latex \
                 texlive-collection-fontsrecommended \
                 texlive-cite latexmk texlive-textcase \
                 texlive-engrec texlive-parskip \
                 texlive-minted \
                 texlive-europecv \
                 texlive-hyphen-english texlive-hyphen-slovak \
                 texlive-titlesec
\end{minted}


\section{Generovanie \LaTeX dokumentov}

Pre priame spúšťanie z príkazového riadku odporúčame použiť príkaz {\tt latexmk}, ktorý slúži na zostavenie  \LaTeX dokumentov. Príklad použitia je nasledovný:

\begin{minted}{bash}
$ latexmk -pdf -bibtex -shell-escape thesis
\end{minted}

Ak nechcete spúšťať tento príkaz zakaždým po vykonaní zmien v zdrojových súboroch, pridajte programu {\tt latexmk} prepínač {\tt -pvc}, ktorý zabezpečí ich sledovanie a znovuzostavenie výstupu automaticky:

\begin{minted}{bash}
$ latexmk -pdf -bibtex -shell-escape -pvc thesis
\end{minted}

Ak budete chcieť vyčistiť vygenerované výstupy, stačí nástroj {\tt latexmk} spustiť s prepínačom {\tt -c}:

\begin{minted}{bash}
$ latexmk -c thesis
\end{minted}
