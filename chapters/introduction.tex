% !TEX root = ../thesis.tex

\chaptermark{Úvod}
\phantomsection
\addcontentsline{toc}{chapter}{Úvod}

\chapter*{Úvod}


Niekedy si to neuvedomujeme, ale predikcie sú veľkou súčasťou nášho života. Každý deň sa rozhodujeme podľa kombinácie nášho inštinktu a predošlých skúseností zo života, podľa ktorých vydedukujeme pravdepodobnosť úspechu našej akcie. 
\\ \\
Pri probléme predikcie víťazného tímu poznáme množstvo príkladov z rôznych kútov kompetetívnych športov alebo hier. Dôvodom pre tento fenomén je jeden z hlavných poháňačov dnešnej doby, a to sú peniaze. Začiatok stávkovania sa dátuje na viac ako pred 2000 rokmi v starovekej Číne. \cite{gamblinghistory}  V dnešnej dobe sa tipovanie rozrástlo o predikcie v boxe, futbale, dostihoch a tak ďalej. No v posledných rokoch prišli na scénu počítačové hry, ktoré zväčšili obzory stávkovania. Ľudia, ktorých šport až tak nezaujímal, sa mohli dostať do sveta profesionálneho e-sportu a vďaka tomuto faktu bolo oslovené nové publikum. Jedným z problémov je stávkovanie pod legálnym vekovým limitom, čo je ale téma na dlhší čas. 
\\ \\
Pri rozhodovaní sa o našom favoritovi nám môžu slúžiť predošlé hry daných tímov, štatistiky máp, pozícií, či už konkrétneho hráča alebo postavy, za ktorú hrá. Problémom je, že sú ich tisíce a my nemáme dostatočnú kapacitu na spracovanie týchto dát. Ak ale spojíme sily s počítačmi, ktoré majú niekoľko násobne väčší úložný priestor ako my ľudia, môžeme dosiahnuť oveľa lepšie výsledky a vyššie cieľe. Pri spojení nebude dôležitý iba úložný priestor a dáta, ale aj ich spracovanie a implementovanie. V takýchto prípadoch je veľmi výhodné použitie umelej inteligencie a práve tejto činnosti sa budeme venovať v tejto práci. 
\\ \\
V analýze bude potrebné naštudovať všetky hry a ich potenciál v predikcii s pomocou umelej inteligencie a následne vybrať jednu z týchto hier. Po výbere hry bude potrebné prejsť všetky potenciálne predošlé pokusy, určiť, ktoré z informácií sú najpotrebnejšie pre určovanie víťaza, ďalej nájsť najoverenejší a najlepší zdroj na tieto informácie a nakoniec vybrať správne prostredie na implementáciu.
\\ \\
Pri syntéze bude potrebné zjednotiť všetky nadobudnuté informácie a čo najefektívnejšie ich implementovať, pričom budeme musieť vytvoriť prototyp, otestovať ho a upraviť vzhľadom na novo nadobudnuté informácie, čo nás dovedie k vytvoreniu nového a finálneho produktu.
\\ \\
Vo vyhodnotení budeme rozoberať informácie, ktoré nás obohatili od začiatku práce. Efektívnosť finálneho produktu a jeho postupného vývoja bude hlavnou témou. Zároveň odsledujeme porovnanie spokojnosti prvých zákazníkov a ich postrehov.




\section*{Formulácia úlohy}

Úloha spočíva v prvom rade v nájdení najlepšie implementovateľnej hry s najväčším potenciálom pri využití umelej inteligencie na predpoveď víťaza. Následne vybratie správnych dát. 
Rozhodnutie alebo porovnanie možností spracovania daných dát. Výber prostredia, v ktorom bude návrh na prototyp a vytvorenie prvého prototypu. 
\\ \\
Otestovanie prototypu s potenciálnymi zákazníkmi. Poskytneme im prototyp so scenárom, zaevidujeme ich spôsob vykonania, spokojnosť a vytvoríme anonymný dotazník na získanie ďalších informácií. Bude potrebná implementácia požiadaviek, nastavenie nového prototypu, poprípade vytvorenie finálneho produktu, ktorý bude potrebné znovu otestovať. 

\subsection*{Motivácia}

Potenciál má viacero častí.\\
Prvá časť spočíva v rozšírení obzorov použitia umelej inteligencie pri predpovedi e-sport zápasov. Prácou môže byť poukázané na efektívnosť a presnosť pri predikcii, zároveň vytvorenie prototypu a produktu môže ukázať lepšiu cestu pre podobné práce v budúcnosti.
\\ \\
Druhá časť spočíva v prípadnom poskytnutí softvéru profesionálnym tímom, ktoré môžu produkt využiť na zlepšenie stratégie ich tímu, to by mohlo následne výrazne ovplyvniť ich budúce výsledky. Zlepšenie výsledkov je všetko, o čo ide profesionálnym tímom, a to by produkt mohlo zmeniť na veľmi lukratívnu ponuku pre vysokú scénu e-sport atlétov.
\\ \\
Hlavná časť, ale zároveň aj najkontroverznejšia, je poskytnutie softvéru stávkarom, ktorí by s jeho pomocou mohli zlepšiť presnosť ich predpovedí. Ak by bol tento potenciál naplnený a overený, vplyv na stávkovanie by bol enormný a softvér by bol nevyhnutnou súčasťou stávkovacej scény a s nárastom domény by sa výrazne zdvihla cena poskytovania.






