% !TEX root = ../thesis.tex

\chaptermark{Úvod}
\phantomsection
\addcontentsline{toc}{chapter}{Úvod}

\chapter*{Úvod}


Niekedy si to neuvedomujeme, ale predikcie sú veľkou súčasťou nášho života. Každý deň sa rozhodujeme podľa kombinácie nášho inštinktu a predošlých skúseností zo života, podľa ktorých vydedukujeme pravdepodobnosť úspechu našej akcie.
 
Pri probléme predikcie víťazného tímu poznáme množstvo príkladov z rôznych kútov kompetetívnych športov alebo hier. Dôvodom pre tento fenomén je jeden z hlavných poháňačov dnešnej doby, a to sú peniaze. Začiatok stávkovania sa dátuje na viac ako pred 2000 rokmi v rannom období Grécka, kde ľudia tipovali, kto vyhrá olympijské hry. V dnešnej dobe sa tipovanie rozrástlo o predikcie v boxe, futbale, dostihoch a tak ďalej. No v posledných rokoch prišli na scénu počítačové hry, ktoré zväčšili obzory stávkovania. Luďia ktorých šport až tak nezaujímal sa mohli dostať do sveta profesionálneho e-sportu, vďaka tomuto faktu bolo oslovené nové publikum. Jedným z problémov je stávkovanie pod legálnym vekovým limitom, čo je ale téma na dlhší čas. 

Pri rozhodovaní sa o našom favoritovi nám môžu slúžiť predošlé hry daných tímov, štatistiky máp, pozícií, či už konkrétneho hráča alebo postavu, za ktorú hrá. Problémom je, že sú ich tisíce a my nemáme dostatočnú kapacitu na spracovanie týchto dát. Ak ale spojíme sily s počítačmi, ktoré majú niekoľko násobne väčší úložný priestor, ako my ľudia, môžeme dosiahnuť oveľa lepšie výsledky a vyššie cieľe. Pri spojení nebude dôležitý iba úložný priestor a dáta, ale aj ich spracovanie a implementovanie. V takýchto prípadoch je veľmi výhodné použitie umelej inteligencie a práve tejto činnosti sa budeme venovať v tejto práci. 
.


\section*{Formulácia úlohy}

V analýze bude potrebné naštudovať všetky hry a ich potenciál v predikcii s pomocou umelej inteligencie. Vybrať jednu z týchto hier. Prejsť všetky potenciálne predošlé pokusy. Určiť ktoré z informácií su najpotrebnejšie pre určovanie víťaza. Nájsť najoverenejší a najlepší zdroj na tieto informácie a vybrať správne prostredie na implementáciu.

Pri syntéze bude potrebné zjednotiť všetky nadobudnuté informácie a čo najefektívnejšie ich implementovať.
