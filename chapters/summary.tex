% !TEX root = ../thesis.tex
\chaptermark{summary}
\phantomsection
\chapter{Záver}
\label{summary}
Na začiatku našej práce nám nebolo úplne jasné, ako dosiahneme náš cieľ, no vedeli sme, že chceme zlepšiť šancu na úspešné tipovanie víťazného tímu. Naše obzory a ciele sa ale počas pracovania na tejto práci zmenili a uvedomili sme si nielen potenciál, ale aj viacero obmedzení.
\\
Primárne vnímame 2 hlavné využitia. 
\begin{itemize}
	\item Prvé využitie spočíva pri stávkovaní. Je malé okienko času medzi vybratím šampiónov a začiatkom hry. Ak by sme stihli zadať a oznámkovať kompozície v tomto čase a výsledky by určili 2 rôzne známky, tak máme možnosť staviť na tím, ktorý má lepšiu známku. Ak by sme používali túto techniku počas viacerých zápasov, malo by to byť výnosné, no negarantujeme výsledky, stále je veľa ovplyvňujúcich faktorov, ktoré sa často krát nedajú zarátať.
	\item Druhé využitie by bolo v práci analytika. Tak isto, ako veľké firmy majú analytikov, ktorí spracovávajú veľké množstvo dát na zlepšenie chodu ich firmy, majú aj profesionálne tímy členov, ktorí sa venujú analýze. Títo analytici by mohli využiť náš program na zlepšenie ich pohľadu na hru a zároveň určenie typov kompozícií, ktoré sú efektívne. Áno, analytici majú presnejší pohľad na hru, ale ich myseľ nie je dostatočná na spracovanie stá-tisíc hier, čo ale náš program dokáže.
\end{itemize}


