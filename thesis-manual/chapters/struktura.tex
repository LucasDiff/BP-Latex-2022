% !TEX root = ../thesis-manual.tex

\chapter{Štruktúra záverečnej práce}
\label{ch:struktura}

Záverečná práca sa skladá z týchto častí:

\begin{enumerate}
    \item Predhovor
    \item Abstrakt
    \item Úvod práce
    \item Analytická časť práce
    \item Syntetická časť práce
    \item Vyhodnotenie
    \item Záver
	\item Prílohy
	\begin{itemize}
	    \item Životopis
	    \item Systémová príručka
	    \item Používateľská príručka
	\end{itemize}
\end{enumerate}

Záverečná práca musí obsahovať pôvodné myšlienky vytvorené autorom, nesmie byť len jednoduchým prerozprávaním známych faktov a postupov.

\section{Predhovor}

Predhovor v záverečnej práci nie je povinný. Ak je predhovor v práci uvedený, potom obsahuje dôvody pre
voľbu témy práce a pozadie realizácie práce.

\section{Abstrakt}

Abstrakt je stručný opis obsahu záverečnej práce. Z abstraktu musí byť čitateľovi zrejmé čo autor v práci
riešil (problém), ako to riešil (metódy), k čomu v práci dospel (výsledky) a aké sú prínosy jeho riešenia.

\section{Úvod práce}

Úvod práce stručne opisuje stanovený problém, kontext problému a motiváciu pre riešenie problému. Z úvodu by malo byť jasné, že stanovený problém doposiaľ nie je vyriešený a má zmysel ho riešiť. Súčasťou úvodu práce je formulácia úlohy (samostatná kapitola), v ktorej sú jasne stanovené ciele záverečnej práce na základe problému. V úvode neuvádzajte štruktúru práce, t.j. o čom je ktorá kapitola. Rozsah úvodu je minimálne 2 celé strany (vrátane formulácie úlohy). Jadro práce musí obsahovať analytickú, syntetickú a vyhodnocovaciu časť. Názvy jednotlivých kapitol a členenie jadra je ponechané na autora.
    
\section{Analytická časť práce}

Analytická časť záverečnej práce analyzuje existujúce podobné prístupy k riešeniu stanoveného problému. Autor práce musí uviesť v tejto časti existujúce prístupy a riešenia, pričom musí zaujať stanovisko k týmto prístupom a riešeniam a opísať ich výhody a nedostatky. Prevažne v tejto časti autor používa odkazy na použité zdroje. Autor v analýze nepreberá odseky z cudzích prác ale uvádza prevažne vlastné postoje podložené odkazmi na literatúru. Je odporúčané aby bola analýza podporená aj experimentmi ak to umožňuje téma práce (napr. vyskúšam softvér). Analytická časť tvorí zvyčajne \nicefrac{1}{4} jadra práce.

\section{Syntetická časť práce}

Syntetická časť opisuje metódy použité na syntézu riešenia a opisuje syntézu samotného riešenia (zvyčajne je to návrh/implementácia softvérového resp. hardvérového riešenia), pričom sa opiera o závery analytickej časti práce. Syntetická časť tvorí zvyčajne \nicefrac{1}{2} jadra práce.

\section{Vyhodnotenie}

Vyhodnocovacia časť je kľúčovou časťou záverečnej práce. Tato časť obsahuje vyhodnotenie navrhnutého (vytvoreného) riešenia. Uprednostňované je objektívne vyhodnotenie výsledkov práce, ktoré sa opiera o meranie a štatistické metódy, prípadne matematické dôkazy. V prípade nameraných hodnôt musí autor opísať metódu merania, priebeh merania, výsledky a interpretáciu výsledkov v kontexte riešeného problému a stanovených cieľov. Na základe vyhodnotenia riešenia autor opíše prínosy svojej práce. Vyhodnocovacia časť tvorí zvyčajne \nicefrac{1}{4} jadra práce.

\section{Záver}

Záver práce obsahuje zhrnutie výsledkov práce s jasným opisom prínosov a pôvodných (vlastných) výsledkov autora a vyhodnotenie splnenia stanovených cieľov. Je to stručné zhrnutie informácií uvedených v záverečnej práci. Záver by nemal obsahovať nové informácie. V závere by mal tiež autor poukázať na prípadné otvorené otázky, ktoré sú nad rámec rozsahu práce a mal by odporučiť ďalšie aktivity na pokračovanie pri riešení problému. Rozsah záveru je minimálne 1 celá strana.
