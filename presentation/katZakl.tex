\subsection{Teória kategórií}

\frame
{
  \frametitle{Teória kategórií}

  \begin{block}{Teória kategórií bola formulovaná v r. 1945 autormi}
  	\begin{center}
%\begin{tabular}{cccc}
%	Samuel \vspace{0.1cm} Eilenberg	& \pgfuseimage{eilenberg}	&	\pgfuseimage{maclane} &	Saunders Mac Lane	\\
%\end{tabular}
	\begin{columns}
		\begin{column}{0.23\textwidth}
			\alert{Samuel} \vspace{0.1cm} \alert{Eilenberg}
		\end{column}
		\begin{column}{0.25\textwidth}
			\pgfuseimage{eilenberg}
		\end{column}
		\begin{column}{0.25\textwidth}
			\pgfuseimage{maclane}
		\end{column}
		\begin{column}{0.17\textwidth}
			\alert{Saunders} \vspace{0.1cm} \alert{Mac Lane}
		\end{column}
	\end{columns}
\vspace{-0.27cm}

\structure{\begin{figure}
\setlength{\unitlength}{1mm}
	\begin{picture}(40,11)
	\put(1,5){$Morphs$}
	\put(18,9){$dom$}
	\put(17,7){\vector(1,0){10}}
	\put(17,5){\vector(1,0){10}}
	\put(18,1){$cod$}
	\put(29,5){$Objects$}
  \end{picture}
\end{figure}}
		\end{center}

\begin{itemize}
		\item Objekty sú štruktúry
		\item Morfizmy sú vzťahy medzi štruktúrami
%		\item Každý objekt má identický morfizmus
%		\item Kompozícia morfizmov je morfizmus v kategórii
	\end{itemize}

%\alert{Kategória množín:}
%    \begin{tabular}{ll}
%           \structure{Objekty} 	&	množiny									\\
%	         \structure{Morfizmy}	&	funkcie medzi množinami
%        \end{tabular}
  	\end{block}
}

% kategorický súčin je zovšeobecnením karteziánskeho súčinu v pojmoch teórie kategórií
% Kategoriu tvorí dvojica tried: trieda štruktúr a trieda relácií medzi týmito štruktúrami pričom štruktúry tvoria objekty kategórie a relácie medzi nimi sa nazývajú morfizmy. Pritom je však dôležité uvedomiť si, že objekty kategórie nemusia byť množiny a morfizmy nemusia byť funkcie medzi nimi, teda teória kategórií predstavuje efektívny nástroj na štúdium rôznych matematických štruktúr ktoré presahujú rámec klasickej teórie množín.