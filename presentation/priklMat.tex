\subsection{Matematické vyjadrenie funkcie}
\frame{
	\frametitle{Funkcia vyp - matematické vyjadrenie}
	\begin{block}{Funkcia $vyp(n,v)$ - matematické vyjadrenie}
		\vspace{0.3cm}
		Nech \structure{$n$}, \structure{$v$} sú z množiny celých čisel, potom platí
		
		\alert{\begin{displaymath}
			vyp(n,v)=\left\{ 	\begin{array}{ll}
																		v & \textrm{ak $n=0$}\\
				vyp(n~div~10,~v~*~(n~mod~10)) & \textrm{ináč}
												\end{array} \right.
		\end{displaymath}}
		
		a pri dosadení konkrétnych hodnôt je výpočtová postupnosť rekurzie je nasledovná:
		
		\structure{\begin{displaymath}
			vyp(377,1)=vyp(37,7)=vyp(7,49)=vyp(0,147)=147
		\end{displaymath}}
		
	\end{block}
} 