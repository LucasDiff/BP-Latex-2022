%% -----------------------------------------------------------------
%% This file uses UTF-8 encoding
%%
%% For compilation use following command:
%% latexmk -pdf -pvc -bibtex thesis
%%
%% -----------------------------------------------------------------
%%                                     _     _      
%%      _ __  _ __ ___  __ _ _ __ ___ | |__ | | ___ 
%%     | '_ \| '__/ _ \/ _` | '_ ` _ \| '_ \| |/ _ \
%%     | |_) | | |  __/ (_| | | | | | | |_) | |  __/
%%     | .__/|_|  \___|\__,_|_| |_| |_|_.__/|_|\___|
%%     |_|                                          
%%
%% -----------------------------------------------------------------

\documentclass{kithesis}

% Additional packages
\usepackage[main=slovak,english]{babel}
% For thesis written in English just change the order of languages:
% \usepackage[main=english,slovak]{babel}

\usepackage{listings}  % for source code
% Listings settings
% See for details: https://en.wikibooks.org/wiki/LaTeX/Source_Code_Listings
\lstset{
	basicstyle=\small\ttfamily,  % smaller typewriter font
	showstringspaces=false       % don't show spaces in string
}

% Location of file with bibliography resources
\addbibresource{chapters/bibliography.bib}

% Variables
%\thesisspec{figures/thesisspec.png} 

\title{Prediction of the winning team in profesional esport matches \br }{Predikcia víťazného tímu v profesionálnych e-sports zápasoch \br}

%\author{Janko Hraško}
\author{Lukáš}{Németh}
\supervisor{Róbert Rauch} %veduci prace
%\college{University of Žilina}{Žilinská univerzita} %univerzita
%\faculty{Faculty of Electrical Engineering and informatics}{Fakulta elektrotechniky a informatiky} %fakulta
%\department{Department of Computers and Informatics}{Katedra počítačov a informatiky} %katedra
%\departmentacr{DCI}{KPI} % skratka katedry
%\thesis{Master thesis}{Diplomová práca} %typ prace
\submissiondate{27}{5}{2022}
%\fieldofstudy{9.2.1 Informatika}
%\studyprogramme{Informatika}
%\city{Košice} %mesto
\thesisspec{zadavaci-list.png}
\keywords{Machine learning, prediction, e-sports, team composition}{Strojové učenie, predikcia, e-sports, tímová kompozícia}
%\declaration{som nepodvadzal}

\abstract{%
	% english 
	Our focus in this bachelor thesis was on the use of machine learning in predicting the winner of a profesional e-sports match. In the first chapter we analyzed the potential of three different online games as well as three types of machine learning. In the second chapter we focused on getting data from databases and adjusting the data to readable structures. We then transform and classify this data into different classes. For better visualization we develop a grading system to categorize all the different team compositions. At the end of the second chapter we design a user interface that can be used if we decide to sell our program to clients. In the third chapter we take a better look at the results and it the last chapter we process all the newly gained knowledge and describe the best uses of our program.
}{%
	% slovak 
	V záverečnej práci sa zameriavame na použitie strojového učenia pri určení výhercu v profesionálnych e-sports zápasoch. V prvej kapitole analyzujeme potenciál troch rôznych hier a zároveň tri rôzne spôsoby strojového učenia. V druhej kapitole sa venujeme prevzatiu potrebných dát z databáz a ich spracovaniu. Rovnako riešime úpravu týchto dát a ich klasifikáciu do viacerých tried. Kvôli lepšej prehľadnosti vytvoríme známkovanie tímových kompozícií a zároveň otestujeme získané výsledky. V závery druhej kapitoly navrhujeme používateľské rozhranie, ktoré by mohlo byť využité pri predaji programu zákazníkom. V tretej časti sa bližšie pozrieme na výsledky testovania a v poslednej časti skonštatujeme, aké hlavné využitia náš program poskytuje.
}

\acknowledgment{Na tomto mieste by som rád poďakoval svojmu vedúcemu práce za jeho čas, veľa dobrých pripomienok a rád.
Rovnako by som sa rád poďakoval mojej rodine, ktorá mi umožnila a pomohla plne sa sústrediť na moje štúdium a pracovanie na mojej bakalárskej práci.}

% if you want to work only on selected chapters
%\includeonly{chapters/analyza} %,chapters/synteza}

% Load acronyms
\input{acronyms}


%% -----------------------------------------------------------------
%%          _                                       _   
%%       __| | ___   ___ _   _ _ __ ___   ___ _ __ | |_ 
%%      / _` |/ _ \ / __| | | | '_ ` _ \ / _ \ '_ \| __|
%%     | (_| | (_) | (__| |_| | | | | | |  __/ | | | |_ 
%%      \__,_|\___/ \___|\__,_|_| |_| |_|\___|_| |_|\__|
%%                                                      
%% -----------------------------------------------------------------

\begin{document}
%% Title page, abstract, declaration etc.:
\frontmatter{}

%% List of code listings, if you are using package minted
%\listoflistings

%\pagenumbering{arabic}
%% Chapters
% !TEX root = ../thesis.tex

\chaptermark{Úvod}
\phantomsection
\addcontentsline{toc}{chapter}{Úvod}

\chapter*{Úvod}


Niekedy si to neuvedomujeme, ale predikcie sú veľkou súčasťou nášho života. Každý deň sa rozhodujeme podľa kombinácie nášho inštinktu a predošlých skúseností zo života, podľa ktorých vydedukujeme pravdepodobnosť úspechu našej akcie.
 
Pri probléme predikcie víťazného tímu poznáme množstvo príkladov z rôznych kútov kompetetívnych športov alebo hier. Dôvodom pre tento fenomén je jeden z hlavných poháňačov dnešnej doby, a to sú peniaze. Začiatok stávkovania sa dátuje na viac ako pred 2000 rokmi v rannom období Grécka, kde ľudia tipovali, kto vyhrá olympijské hry. V dnešnej dobe sa tipovanie rozrástlo o predikcie v boxe, futbale, dostihoch a tak ďalej. No v posledných rokoch prišli na scénu počítačové hry, ktoré zväčšili obzory stávkovania. Luďia ktorých šport až tak nezaujímal sa mohli dostať do sveta profesionálneho e-sportu, vďaka tomuto faktu bolo oslovené nové publikum. Jedným z problémov je stávkovanie pod legálnym vekovým limitom, čo je ale téma na dlhší čas. 

Pri rozhodovaní sa o našom favoritovi nám môžu slúžiť predošlé hry daných tímov, štatistiky máp, pozícií, či už konkrétneho hráča alebo postavu, za ktorú hrá. Problémom je, že sú ich tisíce a my nemáme dostatočnú kapacitu na spracovanie týchto dát. Ak ale spojíme sily s počítačmi, ktoré majú niekoľko násobne väčší úložný priestor, ako my ľudia, môžeme dosiahnuť oveľa lepšie výsledky a vyššie cieľe. Pri spojení nebude dôležitý iba úložný priestor a dáta, ale aj ich spracovanie a implementovanie. V takýchto prípadoch je veľmi výhodné použitie umelej inteligencie a práve tejto činnosti sa budeme venovať v tejto práci. 
.


\section*{Formulácia úlohy}

V analýze bude potrebné naštudovať všetky hry, ich potenciál v predikcii s pomocou umelej inteligencie a následne vybrať jednu z týchto hier. Po výbere hry bude potrebné prejsť všetky potenciálne predošlé pokusy, určiť, ktoré z informácií sú najpotrebnejšie pre určovanie víťaza, ďalej nájsť najoverenejší a najlepší zdroj na tieto informácie a nakoniec vybrať správne prostredie na implementáciu.

Pri syntéze bude potrebné zjednotiť všetky nadobudnuté informácie a čo najefektívnejšie ich implementovať.

% !TEX root = ../thesis.tex

\chapter{Analýza}

\section{Analýza potenciálnych počítačových hier}
Bližšie si pozrieme kandidátov hier na základe možností dát, produktivity, veľkosti ich publika a investovaných fondov do tipovania.

\subsection{Counter-Strike: Global Offensive}
Hra s priemerným publikom viac ako 500 000 hráčov každý deň, publikovaná v roku 2012. Práve prebieha svetový šampionát vo Švédsku s výherným rozpočtom 2 miliónov dolárov. Veľkou časťou tejto hry je verejne dostupný trh s predmetmi získateľnými iba cez túto hru. Tieto predmety majú na podobnom základe ako NFT´s peňažnú hodnotu a poskytujú hráčom vsadiť cez tretie strany na výhercov konkrétnych zápasov, čo viacnásobne zväčšuje tipovací trh. Problémom pri FPS hrách je ale určenie potrebných dát na predpoveď. Výsledok má príliš veľa variácií, kde len milimetre rozhodujú o úplnej zmene priebehu. Dalo by sa určiť víťaza na základe predošlých stretnutí daných dvoch tímov, no tímy sú veľmi ovplyvnené zmenami hráčov, ktoré môžu úplne zmeniť tímovú atmosféru a dáta z predošlých hier sú nepotrebné.
 \subsubsection{Predpovede počas zápasu}
 Jedna hra sa delí na 30 kôl. Prvý tím ktorý dosiahne 16 bodov, vyhráva. Počas kola už firma Valve, tvorca tejto hry, ukázala ich prieskumy a ukazuje, aká je šanca, že daný tím vyhraje kolo na základe miliárd kôl z predchádzajúcich hier. Percentá sa menia podľa počtu živých hráčov, počtu peňazí, mapy, strany a položenia bomby. Za túto funkciu som musel zaplatiť 0,85 eur za mesiac.
  \subsubsection{Predstavenie podobných prístupov}
  Nasledujúce obrázky sú z pozorovania podobných prístupov.
 \begin{figure}
 
 	\includegraphics[width=.9\textwidth]{figures/jednanula}
 	\centering
 	\caption{\LaTeX{} Na začiatku môžeme vidieť 51 percentnú pravdepodobnosť na výhru, pri prvom úmrtí nášho spoluhráča šanca klesla na 36 percent, pri druhom až na 20, ale pri zabití nepriateľa šanca stúpla na 30 percent.  \label{jednanula}}
 \end{figure}

  \begin{figure}
\includegraphics[width=.9\textwidth]{figures/sedemnula}
\centering
\caption{\LaTeX{} Vieme vydedukovať, že čím je šanca na výhru vyššia, tým sa šanca na úmrtie spoluhráča takisto zmenšuje. 
\label{sedemnula}}
\end{figure}
\begin{figure}
	\includegraphics[width=.9\textwidth]{figures/sedemstyri}
	\centering
	\caption{\LaTeX{} Hneď prvé percentu je veľmi nízke, len 24 percent, a to kvôli horšiemu vybaveniu/peňažnej situácie nášho tímu.
		\label{sedemstyri}}
\end{figure}
\begin{figure}
	\includegraphics[width=.9\textwidth]{figures/osempat}
	\centering
	\caption{\LaTeX{} Z prvého prechodu si môžete všimnúť stúpnutie šance z 34 na 48 percent, aj keď bol zabitý jeden z našich spoluhráčov. 
		\label{osempat}}
\end{figure}

 \subsubsection{Vyhodnotenie Counter-Strike: Global Offensive}
 Vyhrávajúci tím a zmeny je možné vidieť počas celého daného kola, ale stávky sa berú pred začatím zápasu.
 
Berúc do úvahy potenciál a veľkosť stávkového trhu nie je aktuálne reálne nájsť spôsob predpovedania, a to kvôli kvantám premenných.

\subsection{Dota 2}
Dota je jednou z dvoch najznámejších MOBA hier na svete, s priemerným počtom hráčov viac ako 400 tisíc. Podobne ako Counter-Strike má predmety a verejný trh. Na druhej strane výherný rozpočet sa s ním nedá porovnať, pretože tento rok prekonal 40 miliónov. Umelá inteligencia tejto hre tiež nie je vzdialená. Počítačom sa podarilo nadľudsky zdokonaliť sa a v uzavretých podmienkach porazili aj najlepších ľudských hráčov na svete. Stávkovanie na ľudské tímy je ale iná vec. Dáta sú pri Dote celkom jasné a je možné ich čerpať z OpenDota API na stránke : https://docs.opendota.com/.

\subsection{League of Legends}
Liga Legiend je najrozšírenejšia MOBA hra na svete s viac ako 3 miliónmi každodenných používateľov a viac ako 100 miliónmi mesačných použivateľov. Riot games má podobne ako Dota verejne dostupné informácie pre developerov na stránke : https://developer.riotgames.com/ 

\section{Analýza dostupných spracovaní dát}
\subsection{Supervised learning}
\subsection{Reinforcement learning}

% !TEX root = ../thesis.tex

\chapter{Syntetická časť}
\label{methodology}
V tejto časti si prejdeme konceptuálny návrh riešenia na predikciu víťazného tímu v profesionálnych esport zápasoch a bližšie sa pozrieme na metódy a hodnoty, ktoré budú potrebné pri jednotlivých fázach z návrhu riešenia, a na finálnu tvorbu prototypu. 
\section{Konceptuálny návrh riešenia}

 \begin{figure}[ht!]
	
	\includegraphics[width=.9\textwidth]{figures/navrhriesenia}
	\centering
	\caption{ Konceptuálny návrh riešenia \label{koncept}}
	
\end{figure}

Ako môžeme vidieť na obrázku \ref{koncept}, na návrh riešenia treba spracovať 4 fázy : 

\begin{enumerate}
	\item Prevziať potrebné dáta z hier cez databázy
	\item Spracovať ich do čitateľnej verzie pre umelú inteligenciu
	\item Použiť dáta na natrénovanie modelu
	\item Spracovať výsledky do čitateľnej formy pre zákazníkov
\end{enumerate}

\section{Prevzatie dát z databáz}
V databázach bolo nie len veľké množstvo druhov dát, ale aj spôsobov akým tieto dáta získať. Bola možnosť zobrať buď dáta zo všetkých hier, čiže aj neprofesionálnych alebo čisto hry iba z profesionálnych turnajov. Po dlhom uvažovaní a viacerých skúškach sme sa rozhodli použiť dáta iba z hier profesionálnych turnajov z celého sveta a konkrétne z posledných 5 mesiacov. Dokopy je rôznych inštancií, ktoré berieme do úvahy viac ako 250 000.

\subsection{Finálny druh dát}
Používané dáta budú 4 druhov. Meno šampióna, pozícia v hre, počet hier, miera výhry. Tieto dáta budú v 2 inštanciach relácií a to v relácií pozorovaný šampión vo vzťahu s jeho tímom a druhej vo vzťahu s nepriateľským tímom.

\subsection{Úprava dát}
Kedže databázy nie sú predpripravené na instantné využitie pri umelej inteligencii, tak sme dáta museli prispôsobiť do podoby, ktoré python, konkrétne scikit learn vie načítať. Dát je príliš veľa na ručnú úpravu, preto sme napísali krátky kód v jave, ktorý načíta databázy, každé slovo dá do uvozdoviek, dá medzi ne čiarku a odstráni všetky tabulátory a medzery.

\subsection{Tokenizácia dát}
Vzhľadom na veľké množstvo rôznych šampiónov, sme sa rozhodli ich rozdeliť na určité kategórie podľa ich významu v hre a ich pozícii v tíme.
V hre je 5 pozícií a pre káždú sme vybrali najčastejšie kategórie, do ktorých patria šampióni. Tu sú kategória aj s krátkymi popismi :
\begin{enumerate}
	\item TOP - pozícia na vrchu mapy, väčšinou izolovaná, až do neskorších fáz hry
	 \begin{itemize}
		\item laner - šampión, ktorý si zakladá na ich sile v prvých štádiách hry
		\item splitpusher - šampión, ktorý je rád izolovaný a rýchlo ničí veže
		\item teamfightdamage - šampión, ktorý exceluje v neskorších fázach hry a jeho primárnou priotiou je urobiť čo najväčšie poškodenie nepriateľským šampiónom počas tímového boja
		\item teamfightcc - skratka cc znamená crowd control, myslené ako udržanie nepriateľského šampióna v nehybnom stave, zatiaľ čo ho spoluhráči zabijú
	\end{itemize}
	\item JUNGLE - pozícia, ktorá sa voľne pohybuje po celej mape, zabíja jeho campy(6 rozdelených miest, kde zabíja monštrá za peniaze a obnovia sa dookola za 2 a pol až 5 minút)  a aplikuje "ganky" (kalkulované útoky na časť mapy)
	\begin{itemize}
		\item clearer - jungler(šampión v jungli), ktorý si zakladá na ich rýchlosti zabíjaní campov
		\item ganker - jungler, ktorého hlavnou úlohou je čo najrýchlejšie pripraviť a vykonať, čo navyšší počet gankov
		\item early - jungler, ktorý je najsilnejší v prvýh častiach hry
		\item late - jugnler, ktorý sa snaží nezomierať a pretrpieť do neskorších fáz hry, v ktorých je najsilnejší
		\item teamfightdamage - jungler, ktorý si v neskorších fázach hry zakladá na výške poškodenia počas tímového boja
		\item teamfightcc - jungler, ktorý si v neskorších fázach hry zakladá na možnosti znehybhniť jedného alebo viacerých nepriateľských šampiónov
	\end{itemize}
	\item MID - centrovaná pozícia, blízko ku TOPu a aj BOTu. Význám sa odvíja od plánu ostatných spoluhráčov
	\begin{itemize}
		\item laner - šampión, ktorý si zakladá na ich sile v prvých štádiách hry
		\item splitpusher - šampión, ktorý je rád izolovaný a rýchlo ničí veže
		\item teamfightdamage - šampión, ktorý exceluje v neskorších fázach hry a jeho primárnou priotiou je urobiť čo najväčšie poškodenie nepriateľským šampiónom
		\item Round - teamfightcc - skratka cc znamená crowd control, myslené ako udržanie nepriateľského šampióna v nehybnom stave, zatiaľ čo ho spoluhráči zabijú
	\end{itemize}
	\item Spracovať výsledky do čitateľnej formy pre zákazníkov
	\begin{itemize}
		\item laner - šampión, ktorý si zakladá na ich sile v prvých štádiách hry
		\item splitpusher - šampión, ktorý je rád izolovaný a rýchlo ničí veže
		\item teamfightdamage - šampión, ktorý exceluje v neskorších fázach hry a jeho primárnou priotiou je urobiť čo najväčšie poškodenie nepriateľským šampiónom
		\item Round - teamfightcc - skratka cc znamená crowd control, myslené ako udržanie nepriateľského šampióna v nehybnom stave, zatiaľ čo ho spoluhráči zabijú
	\end{itemize}
\item Spracovať výsledky do čitateľnej formy pre zákazníkov
\begin{itemize}
	\item laner - šampión, ktorý si zakladá na ich sile v prvých štádiách hry
	\item splitpusher - šampión, ktorý je rád izolovaný a rýchlo ničí veže
	\item teamfightdamage - šampión, ktorý exceluje v neskorších fázach hry a jeho primárnou priotiou je urobiť čo najväčšie poškodenie nepriateľským šampiónom
	\item Round - teamfightcc - skratka cc znamená crowd control, myslené ako udržanie nepriateľského šampióna v nehybnom stave, zatiaľ čo ho spoluhráči zabijú
\end{itemize}
\end{enumerate}
\section{Opis metód}
Hlavné metódy pri učení nášho modelu.
\subsection{Supervised learning}
Podkategória umelej inteligencie, ktorá používa daný dataset na natrénovanie modelu.

\subsection{Reinforcement learning}
Časť umelej inteligencie, pri ktorej sa sledujú kroky inteligentných agentov na maximalizáciu efektívnosti dosiahnutia cieľa.

% !TEX root = ../thesis.tex

\chapter{Vyhodnotenie}
\label{evaluation}
Vo vyhodnocovacej časti by sme sa chceli venovať bližšiemu pohľadu na výsledky testovania.

\section{Testovanie}
Pri testovaní sme použili 58 profesionálnych zápasov, to znamená známku sme pridávali 116 rôznym kombináciám. Ani jedna tímová zostava nebola rovnaká. Najnižšie získané skóre bolo 44,9249, na druhej strane najvyššie získané skóre bolo 55,5978. Tím s najnižším pridaným skóre prehral a tím s najvyšším skóre vyhral, čo je na začiatok celkom povzbudzujúce. 
\\Priemerné skóre tímov, ktoré vyhrali, bolo 51,6450, toto skóre je priamo na hranici známky B. 
\\Priemerné skóre tímov, ktoré prehrali, bolo 50,6645, toto skóre je priamo na hranici známky C.
\subsubsection{Rozdiely medzi tímami}
Na obrázku \ref{rozdiel} je znázornené, aký rozdiel bol medzi víťazným tímom a tímom, ktorý prehral. \\Všetky prípady nad rovinou 0 označujú prípady, pri ktorých náš model určil lepšiu známku tímu, ktorý vyhral, pričom najväčší rozdiel bol 9,3176.
\\ Na druhej strane všetky prípady pod nulou označujú inštancie, kedy zvíťazil tím, ktorý náš model určil ako znevýhodnený, tu bol najväčší rozdiel -3,7837.
\\ Priemerný rozdiel bol v súlade s naším modelom a to 0,9804, čo označuje skoro celotriedový rozdiel.
\begin{figure}[h!]
	
	\includegraphics[width=.9\textwidth]{figures/rozdiel}
	\centering
	\caption{ Rozdiel pridaných skóre pri tímoch \label{rozdiel}}
	
\end{figure}

\subsubsection{Známkovanie testovaných tímov}
Pri určovaní známok sme použili tabuľku, o ktorej sme rozprávali v minulej časti. 
\\ \\
Ako môžete vidieť na obrázku \ref{win}, výsledky tímov, ktoré vyhrali, mali všetky prípady, kde tím dostal hodnotenie S+, čo boli 4 tímy. Na druhej strane 1 tím, ktorý vyhral, dostal známku F, konkrétne im bolo pridané skóre 47,9589, čo je na hranici medzi F a E, takisto tím, ktorý porazili, bol ohodnotený známkou D, čo nie je až taký veľký rozdiel a je to predvídateľné.
\\
Väčšina víťazných tímov skončila so známkou B a C, pričom 8 tímov dostalo známku S.
\begin{figure}[h!]
	
	\includegraphics[width=.9\textwidth]{figures/win}
	\centering
	\caption{ Známkovanie víťazných tímov \label{win}}
	
\end{figure}
\\ \\
Na obrázku \ref{lose} vidíme oznámkovanie tímov, ktoré prehrali ich súboj. Najvyššie oznámkovanie bolo S, kde 3 tímy označené druhou najlepšou známkou S prehrali, konkrétne prehrali proti tímom označeným C, B a D. Zhodou okolností súboj tímu S a D bol prípad najväčšieho rozdielu, kde náš model určil nesprávneho víťaza. Rovnako 6 tímov označených známkou A prehralo proti tímom S+, S, S, B, C, D.
\\ 
Najväčšia porcia tímov, ktoré prehrali dostala známku D, C a E, pričom 10 tímov dostalo známku B.

\begin{figure}[h!]
	
	\includegraphics[width=.9\textwidth]{figures/lose}
	\centering
	\caption{ Známkovanie porazených tímov \label{lose}}
	
\end{figure}

\section{Zhodnotenie výsledky testovania}
Výsledky sú lepšie, než sme predpokladali. Je značný rozdiel v tímových kompozíciach, ktorý náš program vie rozoznať.
\\
V testovacej časti sme odhalili, že v 58 prípadoch bolo 31, kde náš program dal víťaznému tímu lepšiu známku, čo reprezentuje 54 percent zo všetkých prípadov.
V 14 prípadoch dostal lepšiu známku tím, ktorý prehral, čo označuje približne 24 percent z celkových prípadov.
V ostatných 13 prípadoch dostali tímy rovnakú známku, čo je približne 22 percent z prípadov.
\\ 
Čo znamená, že ak budeme tipovať na tímy, ktoré majú rozdielne známkovanie, tak máme približne 68,89 percentnú šancu, že náš program dá lepšiu známku tímu, ktorý zápas vyhrá.

% !TEX root = ../thesis.tex
\chaptermark{summary}
\phantomsection
\chapter{Záver}
\label{summary}
Na začiatku našej práce nám nebolo úplne jasné, ako dosiahneme náš cieľ, no vedeli sme, že chceme zlepšiť šancu na úspešné tipovanie víťazného tímu. Naše obzory a ciele sa ale počas pracovania na tejto práci zmenili a uvedomili sme si nielen potenciál, ale aj viacero obmedzení.
\\
Primárne vnímame 2 hlavné využitia. 
\begin{itemize}
	\item Prvé využitie spočíva pri stávkovaní. Je malé okienko času medzi vybratím šampiónov a začiatkom hry. Ak by sme stihli zadať a oznámkovať kompozície v tomto čase a výsledky by určili 2 rôzne známky, tak máme možnosť staviť na tím, ktorý má lepšiu známku. Ak by sme používali túto techniku počas viacerých zápasov, malo by to byť výnosné, no negarantujeme výsledky, stále je veľa ovplyvňujúcich faktorov, ktoré sa často krát nedajú zarátať.
	\item Druhé využitie by bolo v práci analytika. Tak isto, ako veľké firmy majú analytikov, ktorí spracovávajú veľké množstvo dát na zlepšenie chodu ich firmy, majú aj profesionálne tímy členov, ktorí sa venujú analýze. Títo analytici by mohli využiť náš program na zlepšenie ich pohľadu na hru a zároveň určenie typov kompozícií, ktoré sú efektívne. Áno, analytici majú presnejší pohľad na hru, ale ich myseľ nie je dostatočná na spracovanie stá-tisíc hier, čo ale náš program dokáže.
\end{itemize}






% good linebraking of bibtex url
\setcounter{biburllcpenalty}{7000}
\setcounter{biburlucpenalty}{8000}

%% The bibliography
\printbibliography[heading=bibintoc]

\label{theend} % the last page of the thesis

% List of acronyms
\printglossary[type=\acronymtype,title={\acrlistname}]

% Glossaries
\printglossary
\
%% Appendix
% !TEX root = ../thesis.tex

\chapter*{\appendixlistname}
\addcontentsline{toc}{chapter}{\appendixlistname}

\begin{description}
	\item[\appendixname{} A] Systémová príručka
	\item[\appendixname{} B] Používateľská príručka
	\item[\appendixname{} C] CD médium 
	\\- doc - záverečná práca v čitateľnom formáte(PDF)
	\\- tex - zdrojové súbory záverečnej práce (Latex)
	\\- src - zdrojové kódy riešenia, používané dáta a výsledky v excel tabuľke
\end{description}

\appendix
\renewcommand\chaptername{\appendixname}
% !TEX root = ../thesis.tex

\chapter{Systémová príručka}
Táto časť slúži na opis priečinkov a rozdelenia kódu.
\section*{Rozdelenie kódu}
Funkčný kód je rozdelený na dve časti. Prvá časť sa venuje základnej úprave dát do čitateľnej formy, konkrétne vo funkcii main napísanej v kóde java. Druhá časť, ktorá klasifikuje všetky dáta, spracováva výsledky a vypočítava skóre, je napísaná v Pythone, konkrétne v main funkcii predict.py. 
\section*{Opis kódu}
Prvá časť v kóde main.java má dve funkcie. Je to funkcia GetCharFromString, ktorá vracia aktuálny charakter v stringu a funkcia main, v ktorej otvoríme ako BufferedReader textový súbor 'menochampiona'+1.txt a zároveň vytvoríme nový textový súbor 'menochampiona'+.txt. Následne prechádzame celý textový súbor a upravujeme ho do čitateľnej podoby.
\\ Druhá čast v kóde predict.py má dve hlavné priority. Prvá je zápis všetkých získaných textových súborov z programu javy do jedného dataframu a zároveň ich klasifikácia. A druhá pracuje s daným dataframom a počíta skóre.
\section*{Opis priečinkov}
V priečinku transformdata sa nachádza 332 súborov, z čoho je viac než 320 textových súborov s dátami o každom jednom šampiónovi. Zároveň je tam aj hlavná časť kódu a to súbor predict.py, ktorý má v sebe spracovanie a úpravu všetkých dát. Hlavný súbor s dátami sa volá winratedata.txt ktorý v sebe drží hodnoty všetkých klasifikovaných dát pre každého šampióna. Takisto v hlavnom priečinku môžete nájsť vysledky.xlsx, čo je excelovská tabuľka všetkých otestovaných zápasov. V priečinku src nájdeme kód main.java, pomocou ktorého sme dáta dostali do čitateľnej podoby pre náš python kód. 

\chapter{Používateľská príručka}
V tejto časti si ukážeme ako spustiť a obsluhovať aktuálny kód.

\section*{Potrebné súbory}
Na úspešné odskúšanie kódu na predikciu potrebujete mať textové súbory winratedata.txt, winlose.txt, teams.txt a zároveň samotný súbor s kódom predict.py, ktorý sa nachádza v hlavnom priečinku spolu s textovými súbormi.

\section*{Príprava prostredia a knižníc} 
Odporúčané prostredie je Spyder, ideálne verzia 5.0.5 a novšie. Prostredie Spyder je možné stiahnuť na ich domácej stránke zadarmo. Potrebná knižnica na stiahnutie je pandas, čo môžete vykonať príkazom - pip install pandas, napísaným v konzole.

\section*{Popis používania}
Kód predict.py je potrebné otvoriť v predpripravenom prostredí. Následne na vrchu kódu je 5 premenných top, jg, mid, bot a sup. Pri predikcii kompozície, ktorú chcete odskúšať do týchto premenných, vložte mená šampiónov bez medzier v ich mene s malými písmenami. Ak ste tak vykonali, spustite kód zelenou šípkou a v konzole by mali byť vypísané premenné totalgames, totalscore a známka, čo označuje aktuálnu kompozíciu, ktorú ste zapísali. 


% zivotopis autora
%\curriculumvitae\protect
%Táto časť\/ je nepovinná. Autor tu môže uviesť\/ svoje biografické
%údaje, údaje o~záujmoch, účasti na~projektoch, účasti na~súťažiach,
%získané ocenenia, zahraničné pobyty na~praxi, domácu prax, publikácie
%a~pod.

\end{document}
