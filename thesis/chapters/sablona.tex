\chapter{Práca so šablónou}

\section{Štruktúra projektu}

Projekt záverečnej práce má nasledovnú štruktúru:

\begin{verbatim}
.
|-- CHANGELOG.md
|-- chapters
|   |-- motivacia.tex
|   |-- analyza.tex
|   |-- navrh.tex
|   |-- implementacia.tex
|   |-- vyhodnotenie.tex
|   `-- zaver.tex
|-- figures
|   |-- foto.png
|   `-- tugboat.png
|-- kithesis.cls
|-- thesis.bib
`-- thesis.tex
\end{verbatim}

Význam jednotlivých súborov a priečinkov je nasledovný:

\begin{itemize}
    \item priečinok {\tt \bf{chapters/}} obsahuje {\tt .tex} súbory reprezentujúce samostatné kapitoly záverečnej práce. Ak niektorú z nich chcete do práce vložiť, môžete použiť príkaz \mint{latex}|\include{chapters/nazov.kapitoly}| V prípade, že chcete pracovať len na jednej alebo niektorých kapitolách a nechcete znovu generovať celú prácu, môžete využiť príkaz \mint{latex}|\includeonly{kapitola1,kapitola2}|
    \item priečinok {\tt \bf{figures/}} obsahuje zoznam obrázkov, ktoré boli v práci použité
    \item v súbore {\tt \bf{kithesis.cls}} sa nachádza samotná šablóna \emph{kpithesis}
    \item súbor {\tt \bf{thesis.bib}} obsahuje zoznam literatúry vo formáte \emph{BibTeX}
    \item súbor {\tt \bf{thesis.tex}} predstavuje hlavný súbor záverečnej práce
\end{itemize}


\section{Príkazy na nastavenie vlastností šablóny}

V šablóne je zadefinovaných niekoľko špeciálnych príkazov, pomocou ktorých je možné nastaviť niekoľko vlastností výsledného dokumentu, ako napr. meno autora, názov univerzity, vlastné znenie poďakovania a pod.

\subsection{Príkaz {\tt \textbackslash{}author}}

Pomocou tohto príkazu je možné nastaviť meno a priezvisko autora záverečnej práce. Jeho použitie je ilustrované v nasledovnom výpise kódu:

\begin{listing}[ht]
\begin{minted}[frame=lines]{latex}
\author{Juraj Matta}
\end{minted}
\caption{Nastavenie mena a priezviska autora záverečnej práce}
\end{listing}


\subsection{Príkaz {\tt \textbackslash{}title}}

Pomocou tohto príkazu je možné nastaviť názov záverečnej práce. Jeho použitie je ilustrované v nasledovnom výpise kódu:

\begin{listing}[ht]
\begin{minted}[frame=lines]{latex}
\title{Krokovanie a ladenie programov v jazyku C}
\end{minted}
\caption{Nastavenie názvu záverečnej práce}
\end{listing}

V prípade, že sa vám názov zalomí na titulnej strane nevhodne, môžete ho zalomiť sami napríklad takto:

\begin{listing}[ht]
\begin{minted}[frame=lines]{latex}
\title{Krokovanie a ladenie programov\\ v jazyku C}
\end{minted}
\caption{Nastavenie názvu záverečnej práce cez dva riadky}
\end{listing}


\subsection{Príkaz {\tt \textbackslash{}college}}

Pomocou tohto príkazu je možné nastaviť názov univerzity, kde sa záverečná práca píše. Názov univerzity bude použitý na titulnej, ako aj na druhej strane záverečnej práce.

Tento príkaz má dva parametre:

\begin{enumerate}
    \item {\it názov univerzity} 
    \item {\it mesto}, v ktorom sa univerzita nachádza. 
\end{enumerate}

Použitie tohto príkazu je ilustrované v nasledovnom výpise kódu:

\begin{listing}[ht]
\begin{minted}[frame=lines]{latex}
\college{Zilinska univerzita}{Zilina}
\end{minted}
\caption{Nastavenie názvu univerzity záverečnej práce}
\end{listing}

Pokiaľ však tento príkaz nepoužijete, predvolená hodnota bude použitá {\it Technická univerzita Košice} pre názov univerzity a {\it Košice} pre mesto, v ktorom sa univerzita nachádza.


\subsection{Príkaz {\tt \textbackslash{}faculty}}

Tento príkaz slúži na nastavenie názvu fakulty, na ktorej záverečná práca vznikla. Názov sa následne použije na titulnej a druhej strane záverečnej práce.

Použitie tohto príkazu je ilustrované v nasledovnom výpise kódu:

\begin{listing}[ht]
\begin{minted}[frame=lines]{latex}
\faculty{Hutnicka fakulta}
\end{minted}
\caption{Nastavenie názvu fakulty záverečnej práce}
\end{listing}

Pokiaľ však tento príkaz nepoužijete, predvolená hodnota bude použitá {\it Fakulta elektrotechniky a informatiky}.


\subsection{Príkaz {\tt \textbackslash{}department}}

Pomocou tohto príkazu je možné nastaviť názov katedry, pod hlavičkou ktorej záverečná práca vznikla. Názov katedry sa zobrazí na titulnej, ako aj na druhej strane záverečnej práce.

Tento príkaz má dva parametre:

\begin{enumerate}
    \item {\it názov katedry} 
    \item {\it skratka katedry}
\end{enumerate}

Použitie tohto príkazu je ilustrované v nasledovnom výpise kódu:

\begin{listing}[ht]
\begin{minted}[frame=lines]{latex}
\department{Katedra kybernetiky a umelej inteligencie}{KKUI}
\end{minted}
\caption{Nastavenie názvu katedry}
\end{listing}

Pokiaľ však tento príkaz nepoužijete, predvolená hodnota pre názov katedry bude použitá {\it Katedra počítačov a informatiky} a skratka bude nastavená na hodnotu {\it KPI}.


\subsection{Príkaz {\tt \textbackslash{}supervisor}}

Pomocou tohto príkazu je možné nastaviť meno a priezvisko vedúceho záverečnej práce. Jeho použitie je ilustrované v nasledovnom výpise kódu:

\begin{listing}[ht]
\begin{minted}[frame=lines]{latex}
\supervisor{Leslie Lamport}
\end{minted}
\caption{Nastavenie mena a priezviska vedúceho práce}
\end{listing}


\subsection{Príkaz {\tt \textbackslash{}consultant}}

Pomocou tohto príkazu je možné nastaviť meno a priezvisko konzultanta záverečnej práce. Jeho použitie je ilustrované v nasledovnom výpise kódu:

\begin{listing}[ht!]
\begin{minted}[frame=lines]{latex}
\consultant{Donald E. Knuth}
\end{minted}
\caption{Nastavenie mena a priezviska konzultanta práce}
\end{listing}


\subsection{Príkaz {\tt \textbackslash{}fieldofstudy}}

Pomocou tohto príkazu je možné nastaviť štúdijný odbor. Jeho použitie je ilustrované v nasledovnom výpise kódu:

\begin{listing}[ht!]
\begin{minted}[frame=lines]{latex}
\fieldofstudy{1.2.3 Programovanie}
\end{minted}
\caption{Príklad použitia príkazu {\tt \textbackslash{}fieldofstudy} pre nastavenie štúdijného oboru}
\end{listing}

Ak typ štúdijného oboru nenastavíte, použije sa predvolená hodnota {\it 9.2.1. Informatika}.


\subsection{Príkaz {\tt \textbackslash{}studyprogramme}}

Pomocou tohto príkazu je možné nastaviť štúdijný program. Jeho použitie je ilustrované v nasledovnom výpise kódu:

\begin{listing}[ht!]
\begin{minted}[frame=lines]{latex}
\studyprogramme{Programovanie}
\end{minted}
\caption{Príklad použitia príkazu {\tt \textbackslash{}studyprogramme} pre nastavenie štúdijného programu}
\end{listing}

Ak štúdijný program nenastavíte, použije sa predvolená hodnota {\it Informatika}.


\subsection{Príkaz {\tt \textbackslash{}thesis}}

Pomocou tohto príkazu je možné nastaviť typ záverečnej práce, ako napr. {\it Bakalárska práca} alebo {\it Dizertačná práca}. Jeho použitie je ilustrované v nasledovnom výpise kódu:

\begin{listing}[ht!]
\begin{minted}[frame=lines]{latex}
\thesis{Diplomova praca}
\end{minted}
\caption{Nastavenie typu záverečnej práce}
\end{listing}

Ak typ záverečnej práce nešpecifikujete, použije sa automaticky hodnota {\it Bakalárska práca}.


\subsection{Príkaz {\tt \textbackslash{}declaration}}

Pomocou tohto príkazu je možné nastaviť text čestného vyhlásenia autora záverečnej práce. Príklad nastavenia vlastného textu čestného vyhlásenia je nasledovný:

\mint{latex}|\declaration{Cestne skautske, ze som celu pracu napisal sam.}|

Ak text čestného vyhlásenia nenastavíte, použije sa automaticky hodnota {\it Vyhlasujem, že som záverečnú prácu vypracoval(a) samostatne s~použitím uvedenej odbornej literatúry.}


\subsection{Príkaz {\tt \textbackslash{}submissiondate}}
\subsection{Príkaz {\tt \textbackslash{}abstract}}
\subsection{Príkaz {\tt \textbackslash{}keywords}}
\subsection{Príkaz {\tt \textbackslash{}acknowledgment}}


%%%%%%%%%%%%%%%%%%%%%%%%%%%%%%%%%%%%%%%%%%%%%%%%%%%%%%%%%%%%%%%%%%%%%%%%%%%%%%%%
\section{Príkazy na vygenerovanie špeciálnych strán}

\subsection{Príkaz {\tt \textbackslash{}frontmatter}}
\subsection{Príkaz {\tt \textbackslash{}frontpage}}
\subsection{Príkaz {\tt \textbackslash{}abstractpage}}
\subsection{Príkaz {\tt \textbackslash{}declarationpage}}
\subsection{Príkaz {\tt \textbackslash{}acknowledgmentpage}}
\subsection{Príkaz {\tt \textbackslash{}titlepage}}



